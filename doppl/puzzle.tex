%!TEX root =../mapp-challenge-18-game-book.tex
% ^ leave for LaTeXTools build functionality

\phChapterWorksheet{Main Puzzle ?}{Doppl's Unending Enigma}

\newcommand{\mappDoppl}{D}
\newcommand{\mappDopplOmega}{D{\small{}D}{\footnotesize{}D}{\scriptsize{}D}{\tiny{}D}..}

It's time to put your training to the test by trying to befriend \textbf{Doppl}!
This Ordinary-Type \mappMobimon{} multiplies very quickly, so it's
considered to be quite hard to Wrangle. As you might expect, the natural
numbers, known as the \textbf{finite ordinals} to \mappMobimon{} Wranglers,
are used to account for a finite amount of Doppl.

\(1=\) \mappDoppl

\(2=\) \mappDoppl\mappDoppl

\(3=\) \mappDoppl\mappDoppl\mappDoppl

\(4=\) \mappDoppl\mappDoppl\mappDoppl\mappDoppl

However, the problem is that \textbf{infinite} groups of Doppl are quite
common out in the wild!

\(\omega=\) \mappDoppl\mappDoppl\mappDoppl\mappDoppl\mappDoppl\mappDoppl
\mappDoppl\mappDoppl\mappDoppl\mappDoppl\mappDoppl\mappDoppl\mappDoppl\mappDoppl
\mappDoppl\mappDoppl\mappDoppl\mappDoppl\mappDoppl\mappDoppl\mappDoppl\mappDoppl
\mappDoppl\mappDoppl\mappDoppl\mappDoppl\mappDoppl\mappDoppl\mappDoppl\mappDoppl
\mappDoppl\mappDoppl\mappDoppl\mappDoppl\mappDoppl\mappDoppl\mappDoppl\mappDoppl
\mappDoppl\mappDoppl\mappDoppl\mappDoppl\mappDoppl\mappDoppl\mappDoppl\mappDoppl
\mappDoppl\mappDoppl\mappDoppl\mappDoppl\mappDoppl\mappDoppl\mappDoppl\mappDoppl
\mappDoppl\mappDoppl\mappDoppl\mappDoppl\mappDoppl\mappDoppl\mappDoppl\mappDoppl
\mappDoppl\mappDoppl\mappDoppl\mappDoppl\mappDoppl\mappDoppl\mappDoppl\mappDoppl
\mappDoppl\mappDoppl\mappDoppl\mappDoppl\mappDoppl\mappDoppl\mappDoppl\mappDoppl
\mappDoppl\mappDoppl\mappDoppl\mappDoppl\mappDoppl\mappDoppl

Of course, if they were all the same size, they would quickly run out
of space for their \mappMobimon{} battles. That's
why groups of Doppl will shrink down as necessary to fit into whatever space
is available. Here's a more typical representation of the above group of
Doppl:

\(\omega=\) \mappDopplOmega

This \(\omega\) is the first \textbf{infinite ordinal}, but it's not the last!
You see, bizzare as it sounds, there's always room for another
Doppl to join the party.

\(\omega+1=\) \mappDopplOmega\mappDoppl

\(\omega+2=\) \mappDopplOmega\mappDoppl\mappDoppl

\(\omega+3=\) \mappDopplOmega\mappDoppl\mappDoppl\mappDoppl

So \(\omega+1,\omega+2,\omega+3\) are the next three infinte ordinals.
And yes, Wrangers have reported Doppl groups like these as well.

\(\omega+\omega=\) \mappDopplOmega\mappDopplOmega

\(\omega+\omega+\omega+5=\omega\cdot 3 + 5=\) \mappDopplOmega\mappDopplOmega\mappDopplOmega\mappDoppl\mappDoppl\mappDoppl\mappDoppl\mappDoppl

\(\omega+\omega+\omega+\cdots=\omega\cdot\omega=\omega^2=\)
\mappDopplOmega\mappDopplOmega\mappDopplOmega\mappDopplOmega\mappDopplOmega
\mappDopplOmega\mappDopplOmega\mappDopplOmega\mappDopplOmega\mappDopplOmega
\mappDopplOmega\mappDopplOmega\mappDopplOmega\mappDopplOmega\mappDopplOmega
\mappDopplOmega\mappDopplOmega\mappDopplOmega\mappDopplOmega\mappDopplOmega
\mappDopplOmega\mappDopplOmega\mappDopplOmega\mappDopplOmega\mappDopplOmega
\mappDopplOmega\mappDopplOmega\mappDopplOmega\mappDopplOmega\mappDopplOmega
\mappDopplOmega\mappDopplOmega\mappDopplOmega\mappDopplOmega\mappDopplOmega
\mappDopplOmega\mappDopplOmega\mappDopplOmega\mappDopplOmega\mappDopplOmega


After the discovery of Doppl, the \mappMobimon{} community noticed that
\textbf{ordinal arithmetic} is a little different than what you might be used
to, at least when there are infinite ordinals around, since it is calculated
by spacing two groups of Doppl one after another.

\(3+4=\) \mappDoppl\mappDoppl\mappDoppl{}
\mappDoppl\mappDoppl\mappDoppl\mappDoppl{} =
\mappDoppl\mappDoppl\mappDoppl\mappDoppl\mappDoppl\mappDoppl\mappDoppl{}
\(=7\)

\(7+\omega=\)
\mappDoppl\mappDoppl\mappDoppl\mappDoppl\mappDoppl\mappDoppl\mappDoppl{}
\mappDopplOmega{} =
\mappDopplOmega{} =
\(\omega\)

\((\omega\cdot4+3)+(\omega\cdot2+5)=\)
\mappDopplOmega\mappDopplOmega\mappDopplOmega\mappDopplOmega\mappDoppl\mappDoppl\mappDoppl{}
\mappDopplOmega\mappDopplOmega\mappDoppl\mappDoppl\mappDoppl\mappDoppl\mappDoppl

\hspace{1.5in} =
\mappDopplOmega\mappDopplOmega\mappDopplOmega\mappDopplOmega\mappDopplOmega\mappDopplOmega\mappDoppl\mappDoppl\mappDoppl\mappDoppl\mappDoppl{} =
\(\omega\cdot6+5\)

\newpage

\textbf{Ordinal multiplication} occurs when each Doppl in the second
factor splits into a copy of the group of Doppl given by the first factor.

\(3\cdot 4=\)
(\mappDoppl\mappDoppl\mappDoppl)(\mappDoppl\mappDoppl\mappDoppl)(\mappDoppl\mappDoppl\mappDoppl)(\mappDoppl\mappDoppl\mappDoppl) \(=12\)

\((\omega+1)\cdot 2=\)
(\mappDopplOmega\mappDoppl)(\mappDopplOmega\mappDoppl) =
\mappDopplOmega\mappDopplOmega\mappDoppl{} \(=\omega\cdot 2+1\)

\(2\cdot(\omega+1)=\)
(\mappDoppl\mappDoppl)(\mappDoppl\mappDoppl)(\mappDoppl\mappDoppl)(\mappDoppl\mappDoppl)(\mappDoppl\mappDoppl)...(\mappDoppl\mappDoppl) =
\mappDopplOmega\mappDoppl\mappDoppl \(=\omega+2\)

To become a master \mappMobimon{} Wrangler, see if you can convince yourself
that the following ordinal calculations are correct.

\[TODO\]

If so, then you should be able to figure out the message that the Doppl
are trying to tell you on the following page.
\textbf{Report this message to Game Control to solve this puzzle.}

% Include below for aucTeX integration
%%% Local Variables:
%%% mode: latex
%%% TeX-master: "../mapp-challenge-18-game-book"
%%% End:
