%!TEX root =../mapp-challenge-18-game-book.tex
% ^ leave for LaTeXTools build functionality

Here are some guidelines for local campuses on how to prepare for and
run the event.

\phSection{Schedule Template}

\begin{itemize}
  \item 0:00 - Registration
  \item 0:45 - Orientation
  \item 1:00 - Game Begins with Opening Puzzle
  \item 2:00 - Opening Puzzle due
  \item 4:00 - Cryptic/Meta Puzzles released
  \item 5:00 - Game Ends (all solutions due)
  \item 5:30 - Wrap-Up and Awards
  \item 6:00 - Dismissal
\end{itemize}

\phSection{Classroom Space}

A large \textbf{lecture hall} is recommended for running Registration,
Orientation, and the Wrap-Up. Game Control can be stationed there during
the game as well.

Each team should be given a separate \textbf{classroom} so that they may
openly collaborate with teammates without spoiling puzzles for other teams.
It is useful to affix \textbf{printed signs} on each classroom and Game
Control to help players navigate your space, as well as any additional
signage required to get around.

\phSection{Team Supplies}

\textbf{Scissors and tape} should be provided in each classroom.
Some campuses choose to provide simple \textbf{calculators} to each team.
In addition, \textbf{chalk or whiteboard markers} should be provided if
teams will have access to chalkboards or whiteboards in their room.

Note that teams may choose to bring their own additional supplies, such as
smartphones, laptops, cameras, graph paper, colored pencils, and so on. Due to
the wide availability of these items (particularly phones), we discourage
campuses from banning outside equipment. The puzzles are designed so that they
cannot be solved using Google or brute force methods, with one exception.
Programming or spreadsheets may help optimize Bonus Puzzle solutions, which is
why the Bonus Puzzle is only used to break ties.

\phSection{Copies}

All puzzles are designed to be printed/copied in \textbf{grayscale}, both
for the convience of campuses and for accessibility by players.
See below to account for how many copies are needed throughout the game.
It is recommended to print copies for at least
\textbf{two more teams than you expect to participate} as extras, depending
on your access to last-minute copying.

\textbf{Each volunteer} working at \textbf{Game Control} should have a
\textbf{complete copy of the game book in a binder} for their reference.

\phSection{Registration}

\textbf{Each player} should receive a \textbf{packet} containing the
\textbf{Story}, \textbf{Rules}, and \textbf{Code Sheet} pages of the game book.
They should also receive
a \textbf{pencil} and \textbf{notepad} for use during the Opening Puzzle
and the rest of the game.

Some campuses also choose to distribute other giveaways/swag/brochures
at registration. Many bookstores are willing to provide branded disposable
bags to help distribute materials.

Teams should be directed to their assigned classroom where they can
drop off everything except the provided packet, pencil, and notepad.
They should then return to Game Control's lecture hall to await Orientation.

\phSection{Orientation}

The Story and Rules should be reviewed, and any questions from players
should be answered. In particular, boundaries for where players are allowed
to travel during the game should be established.

\phSection{Opening Puzzle}

\textbf{Each team} should receive \textbf{three copies} of the
\textbf{Opening Puzzle} and supporting documents. The deadline for
completing the Opening Puzzle should be clearly communicated.

If you wish for players to \textbf{explore your campus}, then you should
also provide a \textbf{Campus Map} with twenty locations labeled \(1\)-\(20\).
The Opening Puzzle solves to four of those numbers, so players should be
instructed to visit those four locations. For larger destinations, you should
specify where to visit (e.g. the front door). You may choose to either place
a \textbf{volunteer} at each location to distribute a token to each team that
successfully finds it, or an \textbf{envelope} of tokens for teams to claim at
each location. You may choose to have teams pick up a copy of one of the four
\textbf{Main Puzzles} at each location to serve as this token, but these should
not be the only copies of the Main Puzzles given to each team (see below).

Once this puzzle has been reviewed for all players, you may dismiss the players
to begin solving. Once each team completes the puzzle, they should present
their solution and/or tokens to Game Control. In return they should receive
an \textbf{envelope} containing \textbf{three copies each} of
\textbf{Main Puzzles 1-4} and the \textbf{Bonus Puzzle}, including supporting
documents. They should also
receive a \textbf{Scoresheet} to mirror the \textbf{Scoresheet copy} maintained
at Game Control, updated with the results of the Opening Puzzle.

At the deadline for the puzzle, all remaining teams should return to
Game Control to pick up their puzzle envelope and scoresheet, and move
on to the Main/Bonus Puzzles. If additional
volunteers have been organized to facilitate the exploration of campus,
you may choose to dismiss them at this time.

\phSection{Post-Opening Gameplay}

After the Opening Puzzle,
teams can remain in their provided classroom for the duration of the
competition, except to submit solutions to Game Control.

A volunteer should stand at the door of Game Control to ensure at most
one team is allowed in Game Control at all times. Solutions to the
Main Puzzles, Cryptic Puzzles, and the Metapuzzle are all short words/phrases
and may be communicated to Game Control verbally, but may be written out
on paper if clarification is required. The solution to the Hidden Puzzle
is also a short word/phrase and may be communicated verbally or written out,
but Game Control should only confirm the existence of the Hidden Puzzle
explicitly after receiving a correct solution. Players should be asked
which puzzle they are attempting to solve before giving a solution.

As each Main Puzzle is solved, that team should receive a \textbf{packet}
containing \textbf{three copies} of the corresponding \textbf{Cryptic Puzzle},
including supporting documents.
Once each team has solved all four Cryptic Puzzles, they should receive a
\textbf{packet} containing \textbf{three copies} of the \textbf{Metapuzzle},
including supporting documents.
All teams are allowed to pick up an \textbf{envelope} containing all
unclaimed copies of the Cryptic Puzzles and Metapuzzle during the final hour
of the game.

Each time a puzzle is correctly solved, it should be updated by Game Control
on both their copy and the team's copy of the Scoresheet, including a
timestamp.

Recreational teams are allowed to ask for hints at Game Control at any time
for any Main Puzzle, Cryptic Puzzle, or Metapuzzle. Game Control should ask
players to explain the work they've
done thus far, and give a single hint that should help the team make some
amount of progress. Different teams may receive different hints for the same
puzzle depending on their progress.

Competitive teams may receive hints for the Main Puzzles during the final
hour of the game. Game Control should provide as much help as is necessary;
the goal of the Main Puzzles is to expose players to new types of mathematics,
so most teams should solve most of the Main Puzzles by the end of the game.

Teams cannot be given hints for the Bonus Puzzle or Hidden Puzzle.

Each team is allowed three submissions of the Bonus Puzzle. Generally this
puzzle should be judged by Game Control in front of the players to confirm
the validity of the submission. Each submission is recorded on both
Scoresheets, including crossing out a box for an invalid submission.
Only the best submission from each team is used. If the game has ended
with multiple teams in line for Game Control, all submissions for all teams
should be collected as quickly as possible and graded. Teams may not submit
multiple Bonus Puzzle solutions after the game has ended.

\phSection{Food}

Campuses that will be running the event through lunchtime are encouraged to
provide a \textbf{pizza lunch} for players. This lunch should not interrupt the
game; rather, players should be able to grab a bite to eat to have while they
continue to solve puzzles. In addition, \textbf{snacks}
(fruit, granola bars, etc.) and \textbf{drinks} (bottled water) are nice for
players to have access to during the game. Don't forget to provide
appropriate \textbf{plates, cutlery, napkins, and trashbags}.

This food can be distributed at a \textbf{central location near Game Control}
(but not inside Game Control's room).

\phSection{Wrap-Up and Awards}

At the end of the game, teams should straighten up their classrooms before
returning to Game Control for the Wrap-Up. \textbf{Trash bags} may be
provided for this purpose.

Teams should line up outside Game Control until results have been tabulated.
Once all results have been determined, teams may be seated inside Game Control.
\textbf{Solutions} to all puzzles should be projected and reviewed with
all players.

Awards for Recreational/Competitive teams are treated completely separately
if both Leagues are present.
\textbf{Certificates} should be distributed in random order to all teams placing
below 3rd place. A \textbf{3rd Place Certificate/Trophy} is then awarded.
After reminding the 1st place team to be respectful, a
\textbf{2nd Place Certificate/Trophy} is then awarded, followed by the
\textbf{1st Place Certificate/Trophy}. Opportunities for photographs should
be allowed during this process and after dismissal.

After awards are done, teams may be dismissed.

\phSection{Shirts/Theming}

Campuses may choose to provide/sell \textbf{shirts} to volunteers and players,
keeping in mind that volunteers should be identifyable to players by sight.
To this end, we encourage distributing shirts at Wrap-Up, or using different
colors for players/volunteers.

Optionally, you may encourage teams and/or volunteers to
wear school colors/shirts, or to develop a team name/theme fitting the
game's theme.

\phSection{Social Media}

Players/teachers/volunteers should be encouraged to tag \texttt{@MaPPmath}
and \texttt{\#\phEventAbbr} on Twitter with non-spoiler posts/media during
and after the event.

\phSection{ClueKeeper}

The \textbf{ClueKeeper} app will be piloted at select campuses for this
event. All rule changes relevant to the usage of ClueKeeper's
solution submissions and GPS enforcement should be made clear to players
via the app. Each team must have access to an iOS or Android
\textbf{smart device} with the game downloaded to participate.
ClueKeeper may be downloaded from \texttt{cluekeeper.com}.

%%% Local Variables:
%%% mode: latex
%%% TeX-master: t
%%% End:
